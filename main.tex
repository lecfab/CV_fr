
% "ModernCV" CV and Cover Letter LaTeX Template Version 1.1 (9/12/12)
%
% This template has been downloaded from: http://www.LaTeXTemplates.com
% Original author: Xavier Danaux (xdanaux@gmail.com)
% License: CC BY-NC-SA 3.0 (http://creativecommons.org/licenses/by-nc-sa/3.0/)
%
% Important note:
% This template requires the moderncv.cls and .sty files to be in the same
% directory as this .tex file. These files provide the resume style and themes
% used for structuring the document.

%----------------------------------------------------------------------------------------
%	PACKAGES AND OTHER DOCUMENT CONFIGURATIONS
%----------------------------------------------------------------------------------------

\documentclass[11pt,a4paper,roman]{moderncv} % Font sizes: 10, 11, or 12; paper sizes: a4paper, letterpaper, a5paper, legalpaper, executivepaper or landscape; font families: sans or roman
\usepackage[T1]{fontenc}
\usepackage[utf8]{inputenc}
%\usepackage{hyperref}

\moderncvstyle{casual} % CV theme - options include: 'casual' (default), 'classic', 'oldstyle' and 'banking'
\moderncvcolor{orange} % CV color - options include: 'blue' (default), 'orange', 'green', 'red', 'purple', 'grey' and 'black'

\usepackage{lipsum} % Used for inserting dummy 'Lorem ipsum' text into the template

\usepackage[scale=0.75, top=15mm, bottom=2cm, left=14mm, right=13mm]{geometry} % Reduce document margins
%\setlength{\hintscolumnwidth}{3cm} % Uncomment to change the width of the dates column
%\setlength{\makecvtitlenamewidth}{10cm} % For the 'classic' style, uncomment to adjust the width of the space allocated to your name
%----------------------------------------------------------------------
%	NAME AND CONTACT INFORMATION SECTION
%----------------------------------------------------------------------
\usepackage[french]{babel}

\firstname{\textbf{Fabrice}} % Your first name
\familyname{\textbf{Lécuyer}} % Your last name
\title{Docteur en informatique et système complexes du LIP6, Sorbonne Université}
\mobile{+33 (0)6 81 78 29 94}
\email{fabrice.lecuyer@lip6.fr}
\homepage{fabrice.lecuyer.me}{fabrice.lecuyer.me} % first is the url, second is text
%----------------------------------------------------------------------------------------

\begin{document}
\makecvtitle % Print the CV title


\section{Thèse de doctorat (2020 -- en cours)}

\cvitem{Titre}{\emph{Ordonner les nœuds pour passer à l'échelle sur les graphes réels}}
\cvitem{Encadrement}{\link[Lionel Tabourier]{http://lioneltabourier.fr/reseaux.html} \& \link[Clémence Magnien]{https://lip6.fr/Clemence.Magnien/}
}
\cvitem{Description}{Les réseaux générés par les études de différents domaines (biologie, réseaux sociaux, web) sont de plus en plus grands, ce qui impose aux algorithmes de graphes d'avoir une complexité quasi-linéaire pour passer à l'échelle. Ordonner les nœuds du graphe selon des propriétés spécifiques du réseau peut améliorer la vitesse de certains algorithmes, leurs bornes théoriques ou la qualité de leur résultat. Ce projet consiste à implémenter et comparer des ordres existants (comme ceux par degré, dégénérescence ou centralité), et à en définir de nouveaux en exploitant les propriétés des réseaux du monde réel.
}
\cventry{2023}{Manuscrit de thèse : Ordonner les nœuds pour passer à l'échelle sur les grands réseaux réels}{\hfill\break Fabrice Lécuyer}{ [\link[pdf]{https://fabrice.lecuyer.me/public/pdf/Lecuyer_2023_Thesis.pdf}]}{}{}
\cventry{2023}{Séminaires : Ordonner les nœuds pour passer à l'échelle sur les graphes réels}{}{}{}{\link[Invitation]{https://www.di.unimi.it/ecm/home/aggiornamenti-e-archivi/tutte-le-notizie/content/ordering-nodes-to-scale-to-massive-real-world-networks.0000.UNIMIDIRE-102347} en janvier par le \link[Laboratoire d'Algorithmique du Web]{https://law.di.unimi.it} (Université de Milan, Italie), et invitation en mai par \link[Systopia]{https://systopia.cs.ubc.ca} (UBC, Canada).
}
\cventry{2023}{Vertex cover quality certification on real-world networks}{\hfill\break Fabrice Lécuyer, Lionel Tabourier, Clémence Magnien}{soumis}{}{}
\cventry{2023}{Charting mobility patterns in the scientific knowledge landscape}{\hfill\break Chakresh Kumar Singh, Liubov Tupikina, Fabrice Lécuyer, Michele Starnini, Marc Santolini}{soumis [\link[pdf]{https://arxiv.org/abs/2302.13054}]}{}{}
\cventry{2023}{Grands réseaux complexes : mettre de l'ordre dans les triangles}{\hfill\break Fabrice Lécuyer, Louis Jachiet, Clémence Magnien, Lionel Tabourier}{\link[ALENEX]{https://www.siam.org/conferences/cm/conference/alenex23} [\link[pdf]{https://doi.org/10.48550/arXiv.2203.04774}]}{}{Présenté à la conférence \link[FRCCS'22]{https://iscpif.fr/frccs2022/} [\link[résumé]{https://fabrice.lecuyer.me/public/pdf/Lecuyer_2022_FRCCS.pdf}] et au workshop \link[MLG]{http://www.mlgworkshop.org/} de KDD'22.
}
\cventry{2022}{Certifier la qualité d’une heuristique sur des graphes réels}{\hfill\break Fabrice Lécuyer}{non publié [\link[pdf]{https://fabrice.lecuyer.me/public/pdf/Lecuyer_2022_Quality-certification.pdf}]}{}{Présenté à \link[JGA'22]{https://jga2022.sciencesconf.org} [\link[résumé]{https://fabrice.lecuyer.me/public/pdf/Lecuyer_2022_JGA.pdf}] et à la conférence \link[FRCCS'23]{https://iutdijon.u-bourgogne.fr/ccs-france/}.
}
\cventry{2021}{[Replication] Speedup Graph Processing by Graph Ordering}{\hfill\break Fabrice Lécuyer, Maximilien Danisch, Lionel Tabourier}{\link[ReScience]{http://rescience.github.io} [\link[pdf]{https://doi.org/10.5281/zenodo.4836230}]}{}{}

%--------------------------------------------------------------------------
%	EDUCATION SECTION
%--------------------------------------------------------------------------
\section{Formation}
\cventry{2015--2020}{Master d'informatique}{École Normale Supérieure de Lyon}{France}{}{Majeure en informatique théorique : algorithmique, théorie des graphes, probabilités, complexité, algèbre... Mineure en systèmes complexes : réseaux, épidémiologie, thermodynamique, science des données...
}
\cventry{2018--2019}{Année de césure pour voyages et projets personnels}{}{}{}{Voyages sac-au-dos de plusieurs mois pour découvrir d'autres paysages, langues et systèmes politiques, et pour rencontrer des personnes aux coutumes et priorités variées.
}
\cventry{2017--2018}{Erasmus : Licence de physique}{Ludwig Maximilian Universität}{Munich, Allemagne}{}{Majeure : physique théorique et appliquée. Mineure : langue allemande.}
\cventry{2013--2015}{Classes préparatoires MPSI/MP*}{Lycée Saint-Louis}{Paris, France}{}{Préparation intensive en maths, phisique et informatique pour préparer les concours d'entrée en école d'ingéniérie ; admission à l'ENS Lyon.}

%--------------------------------------------------------------------------
%	RESEARCH EXPERIENCE
%--------------------------------------------------------------------------
\section{Expérience académique}
\cventry{2020\\six mois}{Détection de communautés temporelles}{équipe Data Science, LIRIS}{Lyon}{}{Détecter et analyser les communautés dans des réseaux à haute résolution temporelle. Transposer les modèles statiques aux flots de liens. Confronter ces modèles aux nombreux résultats de la litérature. Construire des méthodes qui passent à l'échelle, et les tester sur des données réelles ou synthétiques.
  [\link[report]{https://fabrice.lecuyer.me/public/pdf/Lecuyer_2020_Dynamical-community-detection.pdf}]}
\cventry{2017\\trois mois}{Modélisation d'un réseau hospitalier}{Barabási lab, Northeastern University}{Boston}{}{Modéliser les trajectoires hospitalières à partir des données issues de millions d'hospitalisations. Manipuler différents modèles de mobilité comme la gravité ou la radiation, et les comparer aux données officielles des recensements. Étudier la corrélation entre les groupes démographiques et certaines maladies.
  [\link[report]{https://fabrice.lecuyer.me/public/pdf/Lecuyer_2017_Hospital-network-model.pdf}]}
\cventry{2016\\un an}{Logiciel de communication pour l'aide au handicap}{avec dix élèves de Master}{Lyon}{}{Concevoir un système de communication pour personnes paralysées. Programmer un casque d'électro-encéphalographie, définir des moyens d'expression rapide en collaboration avec des linguistes. Coordonner de nombreuses personnes du milieu médical et participer à deux hackathons.
 }
\cventry{2016\\deux mois}{Réseaux de régulation biologique}{Institut de Cybernétique}{Nantes}{}{Analyser le réseau d'interaction des gènes et les graphes d'inférence théoriques. Organiser des données réelles sur les activations et inhibitions de protéines. Simuler le comportement d'un système impliquant plusieurs gènes.
  [\link[report]{https://fabrice.lecuyer.me/public/pdf/Lecuyer_2016_Reseaux-regulation-biologique.pdf}]}

%----------------------------------------------------------------------------------------
%	WORK EXPERIENCE SECTION
%----------------------------------------------------------------------------------------

\section{Expérience professionnelle}
\cventry{2022--...}{Bénévole pour l'énergie solaire}{Enercitif}{Paris}{}{Participer à l'association qui installe des panneaux solaires sur les toits de Paris. Surveiller les centrales en exploitation et chercher de nouvelles surfaces auprès des acteurs privés ou publics.
}
\cventry{2021--2023}{Membre du Conseil des doctorats}{LIP6}{Paris}{}{Réunions mensuelles avec la direction du laboratoire pour travailler sur les difficultés administratives et scientifiques que rencontrent les personnes en thèse.
}
\cventry{2020--2023}{Enseignement en monitorat}{Sorbonne Université}{Paris}{}{Algorithmique (complexité, itération et récursion, arbres binaires, graphes), programmation (code mathématique enseigné sur une version restreinte de Python).
}
\cventry{2019--2020}{Colles de physique en PCSI/PSI}{Prépa CPE Chartreux}{Lyon}{}{}
\cventry{2017--2018}{Développeur Frontend}{gestion de flotte Carsync, Vispiron GmbH}{Munich}{}{Développer un nouveau produit sous Angular2 avec une équipe de développeurs et développeuses.}
\cventry{2009--2013}{Développeur web en freelance}{jeu en ligne Terre de Feu}{}{}{Concevoir et programmer l'entierté du backend et du frontend. Animer une communauté de centaines de joueurs et joueuses.}

%----------------------------------------------------------------------------------------
%	SKILLS SECTION
%----------------------------------------------------------------------------------------

\section{Autres}
\cvitem{Informatique}{\textbf{c++, python, js, angular, sql, html5, css, php, git, c}, \LaTeX, caml, linux, excel...}
\cvitem{Langues}{\textbf{français (langue natale), anglais (bilingue c2)}, allemand (c1), espagnol (b2).}
\cvitem{Musique}{prof de piano, chanteur classique (Unichor de Munich, Académie de musique de Paris), altiste, accordéoniste.}
\cvitem{Centres d'intérêt}{sobriété énergétique, relations internationales, linguistique...}

%----------------------------------------------------------------------------------------

\end{document}

